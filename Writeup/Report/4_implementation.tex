\chapter{Implementation and Testing}

\section{Introduction}
This chapter describes the system's practical implementation and the techniques used to verify its functionality, building on the architectural and design choices discussed in the preceding chapter. The focus now is on the actual construction and verification of the prototype, including hardware integration, firmware development, and system testing under realistic settings, whereas Chapter 3 concentrated on high-level structure, software logic, and system interactions.

There are three primary sections to the chapter. Both the software and hardware components development is outlined in the Implementation section, which also covers the difficulties encountered and how they were resolved during its development. The methods for assessing system performance, including sensor reading validation, Bluetooth connection stability, and state transition accuracy, are described in the Testing section. The chapter is concluded with a summary that highlights the main findings and lessons discovered that influenced later system improvements.

By bridging the gap between the operational device and the conceptual model, its implementation guarantees that the proposed solution not only satisfies user expectations but also operates dependably under real-world restrictions.

\section{Implementation}
This section describes the hardware integration and software development steps involved in creating the prototype system. The system design and user requirements previously outlined served as a guide for the implementation, guaranteeing that the final prototype had the desired capabilities in terms of connectivity, data gathering, processing, and user interaction.

\subsection{Hardware integration}
The design started with an Arduino-compatible microcontroller containing a built-in LoRa module. To this device, an Arduino-compatible Bluetooth Low-Energy module (HM-10) was connected, to facilitate communication to-and-fro the various Bluetooth capable medical sensors that might be used with the system. In addition to the BT module, a DS3231 RTC module was connected to be able to reliably track time and hence allow for some automation in terms of taking readings from the medical sensors. The rest of the hardware components used in the system were for the purposes of allowing users to interface with the system. These components are the LEDs, the buzzer, the 16x2 LCD screen, and the buttons. All but the buttons are for output purposes in varying forms (audio, and visual). The buttons are how the user will be able to provide input to the system. In combination with the LCD screen, the buttons can be formed into a simple yet powerful user interface. The LEDs and buzzer are secondary interfaces, serving mostly as quick indicators to the user regarding the system state, and allow for easier recognition of certain conditions (e.g. a red flashing LED along with a buzzer tone is a clearer indicator that there is some error with the system, rather than just printing text to the LCD). During development the system needs a USB connection to a computer for power, though it would be trivial to alter it to allow for everyday batteries to power it. A detailed schematic of the final circuit is shown in Figure \ref{fig:circuit_diagram}

\subsection{Firmware Logic}
To develop the firmware, C++ was used in the Arduino development environment. As discussed in the previous chapter, a finite state machine was implemented to control all system operations and clearly separate functionality according to the state it is intended to run in. State transitions are controlled both by active user input and by passivley monitoring conditions of the system (for example BT disconnection/connection, or reading invalid data a certain number of times). Each major state was implemented as a single function. Intentional state change is handled by using a helper function to explicitly call the state function belonging to the desired new state. This was done so that when first entering a state intentionally the initialisation for that state can occur in the helper function, and prevent it re-initialising if the state function gets called repeatedly every time the main Arduino loop runs (multiple times a second). It also makes state transition clear in code, so that somebody else reading the code can clearly differentiate when a state transition must happen. Another helper function was created to allow for easy logging to the serial monitor of the Arduino IDE, allowing for quick and easy debugging as well as general log messages which can remain after the system is completed (separation of log message levels into DEBUG, and other levels. DEBUG messages can be ignored by setting a single boolean variable to false, leaving general INFO, WARN, and other messages in the production version).

State transitions are event driven. Events that can trigger a state change are selection of a menu option, incoming valid data from the BT module, threshold evaluation, and success/failure conditions as in the READING and TRANSMITTING states.

\subsection{Bluetooth Communication}
Serial Bluetooth communication over the UART (Universal Asynchronous Receiver Transmitter) protocol was implemented to send and receive data to and from paired medical sensors. Sending data to the medical sensor is only to acknowledge valid data received, or request a retry of the transmission. To validate the received data, a simple parsing method was used to first determine what type of reading it represents (blood pressure, temperature, or heart rate), which will influence how the rest of the data must be parsed and interpreted. This was done by always sending a prefix corresponding to each vital sign measured, from the simulation app (BP for blood pressure, TEMP for temperature, HR for heart rate). Depending on the prefix and hence the type of reading represented, the data had to be parsed and split up in different ways. For instance, blood pressure contains two integer values representing systolic and diastolic blood pressure, separated by a forward slash '/', while temperature data is a two-digit integer, followed by a decimal point '.' followed by another single digit integer. Heart rate is simply either a two or three digit integer.

\subsection{User Interface Logic}
The user interface logic is menu driven primarily and incorporates the LEDs and buzzer as secondary state indicators. The structure of the menus consists of the current menu option printed on the first line of the LCD, and any button functions printed on the second line, if applicable. The menu options are tightly coupled to the individual states, ensuring that the correct menu is displayed at all times depending on the current state. Menu option selection logic occurs within the state functions. As different events occur causing state changes, the LEDs and buzzer also provide indications of said events. The primary user interface elements are the buttons and the 16x2 LCD screen, which are used to interact with and display menu options. The three buttons' available functions are "Previous", "Select", and "Next" in all menus but the SETUP sub-states, where they instead function as "Decrease", "Confirm", and "Increase" buttons referring to the threshold values that can be set from those sub-states. In states with only one option such as the READING, PROCESSING, and TRANSMITTING states, the buttons are effectively disabled seeing as the operations performed by those states do not depend on any user interaction. The way those states disable user interaction is by only having a single menu option defined. In the menu handling logic, a state with only one option is considered "non-interactable" and just prints the single option to the LCD while not printing the menu button functions on the second line of the LCD.

\section{Testing}
System testing was done to verify the way it responded to both normal and extreme situations, as well as to guarantee that every component operated reliably. Testing was split into two categories: testing of local device behaviour, and testing of cloud data transmission.

\subsection{Local Device Behaviour Testing}
MIT App Inventor 2 was used to create a smartphone application that mimics real-world operating conditions without the need for genuine medical sensors. Vital sign data could be manually entered and sent to the device over Bluetooth thanks to this application. Further below in this subsection are included screenshots of the app in various modes.

A variety of test data were sent via the smartphone app, including:

\begin{itemize}
	\item Valid readings for all vital signs (Blood Pressure, Temperature, Heart Rate)
	\item Values for boundary conditions near defined thresholds.
	\item Malformed strings, missing data fields, and values that are out of range (examples of invalid data).
\end{itemize}

This made it possible to thoroughly check that, given its current setup and status, the device could appropriately parse, assess, and react to incoming input. The logging functions written specifically for this system were used to verify that what was intended to happen on the device was actually what was happening. These logs were essential for observing:

\begin{itemize}
	\item Raw data received via Bluetooth
	\item Internal state transitions and their triggers
	\item System responses such as accepting/rejecting readings, and warnings/errors
\end{itemize}

Transitions between the READING, PROCESSING, and TRANSMITTING states received special attention. Both standard and edge-case workflows were validated, including error recovery in cases when incorrect readings were received repeatedly.

\subsection{Cloud Data Transmission Testing}
After local data processing was successfully validated, focus shifted to confirming that the system could send data to to the cloud. Making sure the system could consistently offload out-of-bounds readings for additional analysis or clinical action and fail gracefully if required to, was the goal.

To assess this functionality's dependability and robustness, a specific testing strategy was used:

\begin{itemize}
	\item \textbf{Trigger Conditions and State Transitions}
	\item[] The system was tested to confirm that readings falling within bounds transitioned to the CONNECTED state without passing through the TRANSMITTING state first, whereas if the reading fell out-of-bounds, there would be a transition to the TRANSMITTING state. The mobile app was used to send various inputs simulating each case.
	\item \textbf{Data Formatting and Transmission}
	\item[] Each transmitted payload was verified to contain the correct parameters formatted correctly (e.g. vital sign type, value, timestamp).
	\item \textbf{Acknowledgement and Retry Handling}
	\item[] The system was programmed to expect a response from the cloud platform indicating successful receipt of the data, failure on malformed or incomplete data, and a timeout timer indicating the platform is cloud possibly unreachable. Each of these three cases was tested to work as expected, by sending different "readings" through the mobile app and observing the response received by the Arduino using the previously mentioned logging functions, and the data received on the cloud platform. To simulate an unreachable network, valid data was sent from the mobile app with the LoRa gateway disconnected.
	\item \textbf{Error Reporting}
	\item[] Transmission success should not bring up any messages, but failure should produce a message on the LCD, LEDs, and a buzzer tone.
	\item \textbf{Data Integrity}
	\item[] Valid data sent from the device and received on the cloud platform was compared to ensure that it was consistent across the entire chain, from the mobile app, to the device, and finally on the cloud platform.
\end{itemize}

The tests conducted showed that the cloud communication system can handle transmission failures gracefully without affecting the performance of the device in real-world scenarios, and it can reliably report abnormal readings.

\section{Summary}
This chapter covered the suggested system's practical implementation, from hardware integration and communication logic to software architecture and user interface design. A state-driven paradigm served as the foundation for the system's implementation, guaranteeing responsive interaction with users and linked sensors as well as dependable transitions between operational modes. Adaptive threshold setup, real-time feedback through buzzers and LEDs, and strong input validation were prioritised.

A specially designed mobile application that simulated a variety of input circumstances was used for testing. The system's capacity to process both expected and incorrect inputs, appropriately handle state transitions, and send out-of-range values to a cloud service for additional evaluation was confirmed by these tests. In order to keep an eye on internal behaviour and confirm dependability under different circumstances, logging tools were extensively used.

With the implementation completed and verified in a controlled setting, the following chapter offers a critical analysis of the outcomes, looks at the system's advantages and disadvantages, and talks about its wider clinical applicability and room for development.
