\chapter{Background or Project Scope}

\section{Introduction}
\paragraph{The healing process does not end after surgery is complete, but instead it is often a long effort for weeks or months post-operation. Furthermore, it is not just the physical health of the patient which must be considered, but also and perhaps equally important their mental health and quality of life moving forward. In recent decades it has become clearer to medical professionals that extended monitoring following operations greatly impacts the results of surgeries in a positive manner, as stated in \cite{d2014defining}. Of course, there are multiple factors affecting the extent to which complications will arise in a patient after receiving surgery, such as their age, previous health issues, or conditions which may develop in parallel to the primary diagnosis even if they are in fact unrelated. In particular, the elderly are quite vulnerable to more complications, as stated by \cite{kare2024post}, specifically referring to elderly patients who underwent hip fracture surgery. Such complications affect quality of life longterm as well as general ability to function normally, not just physical health, degree of recovery, and survivability \cite{kare2024post}.}

\paragraph{The role of the surgeon in the postoperative context is to firstly make sure the patient is receiving the necessary support to sustain a healthy balance in their body, and do whatever they can to avoid the development of and complications that may appear as a result of the procedure. Should any complications arise despite the medical staff's best efforts, it is then the surgeon's responsibility to recognise the signs pointing to the development of said complications and take appropriate actions to quickly and effectively manage them, allowing the patient to recover to their preoperative state eventually \cite{Surwit_Tam_2008}}

\section{Clinical significance of monitoring each vital sign post-surgery}
\paragraph{Vital signs give a rich picture of patient condition post surgery, providing medical staff with the information required to adequately care for the patient. Different measurements have varying significance depending on the stage of recovery the patient finds themselves in, however they should be continuously measured to provide history and an indication of trends to clearly show the progress of the patient's condition. Immediately after surgery measurements must be taken often as that is the most critical stage of the recovery process. In the beginning, vitals such as respiratory rate and blood pressure are vastly more important due to their indication of how the patient is recovering from anesthesia, but once they sufficiently recover from it pulse rate is a better indicator of the volume of blood in the circulatory system of a patient \cite{Surwit_Tam_2008}}

\paragraph{}

\section{Role of IoT in vital sign monitoring}
\section{Communication protocols in IoT-based monitoring}
\section{Commercially available medical devices}
\section{Integration with custom IoT systems}
\section{Summary}
Summarise what was said in this chapter and link with the next chapter.
